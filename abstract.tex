%   Filename    : abstract.tex 
\begin{abstract}
Manual service flow and data management are still two of the most common challenges that are faced by many businesses and institutions, even in today’s digital age. One such institution is the University of the Philippines Visayas - Regional Research Center (RRC), which currently relies on manual and semi-automated processes, using Google Apps for their entire service delivery process. Although functional, this system is inefficient and limits the RRC's potential, posing challenges not only to the staff but to the clients as well. The main objective of this paper is to develop TUKIB, a centralized system to automate the service flow process and data management of the UPV RRC. Additionally, this paper also aims to explore the development and integration of a chatbot using the Rasa framework. The proposed system aims to reduce manual tasks, improve data management, enhance client support, and provide ease for both staff and clients of UPV RRC, enhancing its overall operational efficiency.

%From 150 to 200 words of short, direct and complete sentences, the abstract 
%should be informative enough to serve as a substitute for reading the entire SP document 
%itself.  It states the rationale and the objectives of the research.  
%In the final Special Problem  document (i.e., the document you'll submit for your final defense), the  abstract should also contain a description of your research results, findings,  and contribution(s).

%  Do not put citations or quotes in the abract.

%Suggested keywords based on ACM Computing Classification system can be found at \url{https://dl.acm.org/ccs/ccs_flat.cfm}

\begin{flushleft}
\begin{tabular}{lp{4.25in}}
\hspace{-0.5em}\textbf{Keywords:}\hspace{0.25em} & Workflow Automation, Chatbot, Rasa, Data Management, Service Flow\\
\end{tabular}
\end{flushleft}
\end{abstract}
