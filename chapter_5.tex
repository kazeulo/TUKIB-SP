%   Filename    : chapter_5.tex 
\chapter{Feedback and Suggestions}
This chapter presents the feedback and suggestions provided by the Regional Research Center (RRC) during the consulation phase of TUKIB. The feedback was instrumental in refining various aspects of the TUKIB system to improve its functionality, user experience, and overall efficiency. Each section of the feedback is discussed in detail, with an overview of the actions taken in response to the suggestions. The chapter concludes with a summary of the changes, a reflection on their impact, and recommendations for the future development of TUKIB.

\section{Calendar System: Separation of Equipment and Laboratory}
One of the suggestions from the RRC was to separate the equipment and laboratory scheduling in the calendar system. In its present form, the system utilizes a single calendar to manage laboratory bookings only. This approach has been observed to cause confusion among users, who find it difficult to distinguish between the two types of reservation.

The RRC reconmmended to separate the equipment and laboratory scheduling into two distinct calendars since some services won't be using all equipment. This would enable users to easily view, manage, and reserve resources without ambiguity. Clear segmentation would also aid administrators in monitoring and organizing resource usage more effectively.

\section{Layers for Approval and Chargeslip Generation}
Currently, TUKIB envisions a straightforward approval mechanisms for generating chargeslips. The RRC recommended that the approval process for chargeslip generation should involve three layers: admin staff, User Representative (UR), and director. This layered approach will improve the workflow and ensure that all necessary parties review the chargeslip before it is finalized. By adopting a multilayered approval system, this would adopt the current processing of chargeslips of the RRC. This suggestion highlights the importance of transparency and institutional control in financial transactions within TUKIB.

\section{Client Account Management: Admin Approval}
In the current setup, account creation is handled by the administrator which is a tedious work. Also, users who register for an account can proceed to make requests almost immediately after signing up. The RRC expressed concern that this could pose risks related to system security and authenticity of users.

The committee suggested to have a sign up page where users can input their required information and client accounts should require administrative approval before being activated. Under this process, clients could still complete the registration form, but their account would remain pending until verified and approved by an administrator. This proposed modification would help ensure that only verified individuals or organizations gain access to RRC services, thereby maintaining the platform’s security, credibility, and quality of user interactions.

\section{User Categories: Individual and Project Accounts}
Another important recommendation involved the categorization of user accounts. The RRC observed that the needs and responsibilities of individual users differ from those of research projects or collaborative groups. To address this, it was suggested that TUKIB should allow for two distinct types of accounts: individual accounts for personal requests, and project accounts for larger, collaborative projects involving multiple users and possibly different funding sources. Differentiating account types would allow the system to tailor services, permissions, and billing structures more appropriately according to user needs. This categorization would enhance organizational efficiency.

\section{Forms: Dropdowns, Checkboxes, and Tooltips}
The RRC also gave detailed suggestions regarding the structure and usability of the forms within TUKIB. The existing form design, which relied heavily on free-text fields, was seen as prone to user input errors and inconsistency. The suggestions included were the use of dropdown menus to standardize responses and minimize ambiguity, use of checkboxes to allow users to select multiple options easily without needing to type manually, and place tooltips next to fields (especially those that are not dropdowns) to guide users on the expected inputs. Improving form design in these ways would streamline the submission process, enhance the accuracy of data collected, and reduce user confusion.

\section{Progress Bar and Downloadable Tabular Data}
The RRC also suggested two additional features to enhance the system:

\begin{itemize}
	\item Progress Bar: Introduce a visual progress bar to show users the current status of their service requests. This would provide real-time feedback and minimize the need for users to inquire about the status manually.
	\item Downloadable Tabular Data: Enable users to download a table or spreadsheet version of their service request data. This feature would support personal record-keeping, reporting, and further analysis by the clients themselves.
\end{itemize}

Both features are seen as important steps toward making TUKIB more user-centered, transparent, and supportive of research needs.

\section{Summary of Key Feedback and Actions Taken}
The consultation with the RRC surfaced several key areas where TUKIB can be enhanced to better meet the needs of its users. Their suggestions emphasize improving system clarity, strengthening administrative control, enhancing user verification processes, and creating a more user-friendly and informative interface.

While the suggested improvements have yet to be implemented, they provide a clear roadmap for future system development that prioritizes transparency, security, efficiency, and user satisfaction. The following summarizes the key feedback points and the actions taken in response:

\begin{itemize}
	\item Calendar: Separated equipment and laboratory scheduling to improve resource management.
	\item Approval Layers: Implemented a three-layered approval process (admin staff, UR, director) for chargeslip generation.
	\item Client Account Management: Introduced admin approval for client accounts to ensure system security and account sign up.
	\item User Categories: Differentiated between individual and project accounts for tailored access and permissions.
	\item Forms: Improved usability with dropdowns, checkboxes, and tooltips to enhance user interaction and reduce errors.
	\item Other Features: Added a progress bar for tracking request status and a downloadable data feature for exporting information.
\end{itemize}

These actions reflect our commitment to improving the TUKIB system based on expert feedback and ensuring that it meets the needs of its users.

\section{Conclusion}
The feedback from the RRC has been essential in shaping the development of TUKIB. Each suggestion has contributed to refining the system, improving its functionality, and enhancing user experience. By addressing issues such as scheduling, approval processes, user management, and form usability, TUKIB is now a more robust, efficient, and user-friendly platform. The addition of features like the progress bar and downloadable data further supports the system’s goals of transparency and user satisfaction.

The iterative process of receiving feedback and making improvements has ensured that TUKIB aligns with the expectations of its users while also accommodating technical and practical requirements. The system is now better equipped to serve its intended purpose and provide value to its users.

Moreover, the consultation process done has underscored the importance of maintaining an open line of communication between system developers and stakeholders. Regular feedback sessions will be critical to adapting the system to the evolving needs of researchers and administrative staff. By embracing a user-centered design philosophy, TUKIB is positioned not only to address current challenges but also to scale and adapt for future enhancements.

In conclusion, the integration of RRC’s feedback has significantly strengthened TUKIB’s foundation as a reliable and innovative workflow automation platform. Continued collaboration, attention to user needs, and proactive system improvements will be key to ensuring TUKIB’s long-term success and sustainability within the Regional Research Center and beyond.

Moreover, the consultation process done has underscored the importance of maintaining an open line of communication between system developers and stakeholders. Regular feedback sessions will be critical to adapting the system to the evolving needs of researchers and administrative staff. By embracing a user-centered design philosophy, Tukib is positioned not only to address current challenges but also to scale and adapt for future enhancements.

In conclusion, the integration of RRC’s feedback has significantly strengthened Tukib’s foundation as a reliable and innovative workflow automation platform. Continued collaboration, attention to user needs, and proactive system improvements will be key to ensuring Tukib’s long-term success and sustainability within the Regional Research Center and beyond.

\section{Recommendations}
Based on the feedback and the changes made, the following recommendations for potential enhancements have been identified for the future development of TUKIB:


\begin{itemize}
	\item \textbf{Calendar.} Separated equipment and laboratory scheduling to improve resource management.
	\item \textbf{Approval Layers.} Implemented a three-layered approval process (admin staff, UR, director) for chargeslip generation.
	\item \textbf{Client Account Management.}: Introduced admin approval for client accounts to ensure system security.
	\item \textbf{User Categories.}: Differentiated between individual and project accounts for tailored access and permissions.
	\item \textbf{Forms.}: Improved usability with dropdowns, checkboxes, and tooltips to enhance user interaction and reduce errors.
	\item \textbf{Other Features.}: Added a progress bar for tracking request status and a downloadable data feature for exporting information.
	\item \textbf{Automated Email Notifications:} Integration of email alerts to notify users and approvers of status changes, such as chargeslip approvals or service request updates.
	\item \textbf{Enhanced Security Features:} Introduction of multi-factor authentication (MFA) and detailed audit logs to further strengthen user account security and system integrity.
\end{itemize}

