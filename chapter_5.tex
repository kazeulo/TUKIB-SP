%   Filename    : chapter_5.tex 
\chapter{Feedback and Suggestion}
This chapter presents the feedback and suggestions provided by the Regional Research Center (RRC) during the consulation phase of TUKIB. The feedback was instrumental in refining various aspects of the Tukib system to improve its functionality, user experience, and overall efficiency. Each section of the feedback is discussed in detail, with an overview of the actions taken in response to the suggestions. The chapter concludes with a summary of the changes, a reflection on their impact, and recommendations for the future development of Tukib.

\section{Calendar System: Separation of Equipment and Laboratory}
One of the suggestions from the RRC was to separate the equipment and laboratory scheduling in the calendar system. In its present form, the system utilizes a single calendar to manage laboratory bookings only. This approach has been observed to cause confusion among users, who find it difficult to distinguish between the two types of reservation.

The RRC reconmmended to separate the equipment and laboratory scheduling into two distinct calendars since some services won't be using all equipment. This would enable users to easily view, manage, and reserve resources without ambiguity. Clear segmentation would also aid administrators in monitoring and organizing resource usage more effectively.

\section{Layers for Approval and Chargeslip Generation}
Currently, TUKIB envisions a straightforward approval mechanisms for generating chargeslips. The RRC recommended that the approval process for chargeslip generation should involve three layers: admin staff, User Representative (UR), and director. This layered approach will improve the workflow and ensure that all necessary parties review the chargeslip before it is finalized. By adopting a multilayered approval system, this would adopt the current processing of chargeslips of the RRC. This suggestion highlights the importance of transparency and institutional control in financial transactions within TUKIB.

\section{Client Account Management: Admin Approval}
In the current setup, 

\section{User Categories: Individual and Project Accounts}
The suggestion to differentiate between individual accounts and project accounts was made to better tailor the system’s features to the needs of different user types.

\section{Forms: Dropdowns, Checkboxes, and Tooltips}
The RRC provided several suggestions to improve the forms used in Tukib:

\begin{itemize}
	\item Type of analysis should be presented as a dropdown for better user interaction.
	\item Hazard should be a checkbox to make it easier for users to select multiple options.
	\item Tooltips should be added to non-dropdown fields to guide users and improve form clarity.
\end{itemize}

\section{Forms: Dropdowns, Checkboxes, and Tooltips}
The RRC also suggested two additional features to enhance the system:

\begin{itemize}
	\item A progress bar to visualize the status of service requests.
	\item A feature to allow users to download data in a tabular format.
\end{itemize}

\section{Summary of Key Feedback and Actions Taken}
The following summarizes the key feedback points and the actions taken in response:

\begin{itemize}
	\item Calendar: Separated equipment and laboratory scheduling to improve resource management.
	\item Approval Layers: Implemented a three-layered approval process (admin staff, UR, director) for chargeslip generation.
	\item Client Account Management: Introduced admin approval for client accounts to ensure system security.
	\item User Categories: Differentiated between individual and project accounts for tailored access and permissions.
	\item Forms: Improved usability with dropdowns, checkboxes, and tooltips to enhance user interaction and reduce errors.
	\item Other Features: Added a progress bar for tracking request status and a downloadable data feature for exporting information.
\end{itemize}

These actions reflect our commitment to improving the Tukib system based on expert feedback and ensuring that it meets the needs of its users.

\section{Conclusion}
The feedback from the RRC has been essential in shaping the development of Tukib. Each suggestion has contributed to refining the system, improving its functionality, and enhancing user experience. By addressing issues such as scheduling, approval processes, user management, and form usability, Tukib is now a more robust, efficient, and user-friendly platform. The addition of features like the progress bar and downloadable data further supports the system’s goals of transparency and user satisfaction.

The iterative process of receiving feedback and making improvements has ensured that Tukib aligns with the expectations of its users while also accommodating technical and practical requirements. The system is now better equipped to serve its intended purpose and provide value to its users.

\section{Recommendations}
Based on the feedback and the changes made, the following recommendations are offered for the future development of Tukib:

\begin{itemize}
	\item \textbf{Calendar.} Separated equipment and laboratory scheduling to improve resource management.
	\item \textbf{Approval Layers.} Implemented a three-layered approval process (admin staff, UR, director) for chargeslip generation.
	\item \textbf{Client Account Management.}: Introduced admin approval for client accounts to ensure system security.
	\item \textbf{User Categories.}: Differentiated between individual and project accounts for tailored access and permissions.
	\item \textbf{Forms.}: Improved usability with dropdowns, checkboxes, and tooltips to enhance user interaction and reduce errors.
	\item \textbf{Other Features.}: Added a progress bar for tracking request status and a downloadable data feature for exporting information.
\end{itemize}

