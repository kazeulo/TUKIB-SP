%   Filename    : chapter_2.tex 
\chapter{Review of Related Literature}
\label{sec:relatedlit}

The purpose of this literature review is to provide a comprehensive background on automated systems for workflow automation, especially on service processes, which will inform the development of the system for the University of the Philippines Visayas - Regional Research Center (UPV RRC). This review aims to identify existing solutions, highlight gaps and challenges, and explore technologies that can be used to develop the system to improve the UPV RRC’s operational efficiency.

\begin{comment}

%
% IPR acknowledgement: the contents withis this comment are from Ethel Ong's slides on RRL.
%
Guide on Writing your RRL chapter
 
1. Identify the keywords with respect to your research
      One keyword = One document section
                Examples: 2.1 Story Generation Systems
			 2.2 Knowledge Representation

2.  Find references using these keywords

3.  For each of the references that you find,
        Check: Is it relevant to your research?
        Use their references to find more relevant works.

4. Identify a set of criteria for comparison.
       It will serve as a guide to help you focus on what to look for

5. Write a summary focusing on -
       What: A short description of the work
       How: A summary of the approach it utilized
       Findings: If applicable, provide the results
        Why: Relevance to your work

6. At the end of each section,  show a Table of Comparison of the related works 
   and your proposed project/system

\end{comment}

\section{Challenges in Manual Service Handling}

Manual handling of service processes and data management can often lead to challenges, including inefficiencies, errors, and delays. One of the most common issues is the risk of data entry errors. Even small data entry errors can devastate outcomes, corrupting important data. A study involving three different data entry methods (double entry, visual checking, and single entry) revealed that manual entry, particularly visual checking, has a significantly higher number of errors-2958\% more than double entry methods (Beaty, 1999). These errors can be subtle and difficult to detect, compounding their negative impact on operational efficiency. 

Another limitation of manual service handling is its reliance on human intervention, which frequently results in mistakes that are hard to correct. These errors can escalate operational costs, affect service quality, and lead to customer dissatisfaction. For organizations with manual systems, human error compromises not just data integrity but also the scalability and effectiveness of service operations. Additionally, manual systems lack real-time monitoring capabilities, which are critical for improving service processes. Without automated tracking tools, organizations often miss out on insights that could highlight areas needing improvement.

Current practices in manual service handling also highlight limitations in widely used tools like Google Sheets and Google Docs, which are often insufficient for managing large-scale workflows. These tools lack advanced data retrieval capabilities, and users have reported issues with data not being pulled correctly. According to Okta’s documentation on Google Sheets limitations, there are significant challenges when retrieving and integrating data, leading to inefficiencies in data management processes. Moreover, manual entry in Google Forms is prone to errors, which can undermine the accuracy of collected data.

The impact of these manual methods on stakeholders is substantial. Organizations relying on manual workflows often experience extended processing times, directly affecting service delivery. For instance, tasks that could be automated are unnecessarily prolonged when handled manually, delaying customer satisfaction. Furthermore, manual systems offer limited visibility and tracking capabilities. Without real-time performance metrics, organizations cannot effectively monitor their workflows or identify improvement areas. According to research, companies that automate their workflows experience reduced errors and faster processing times, which lead to improved operational efficiency and better customer outcomes. Thus, the inefficiencies inherent in manual service handling are a barrier to organizational growth, and stakeholders across all levels—from employees to customers—are adversely affected (Davis, n.d.).

\section{Workflow Automation}

\subsection{Workflow automation in different industries}

\section{Client and stakeholder feedback mechanisms}

\section{Existing Systems}

\section{Gaps in the  existing systems and solutions}

\section{Chatbot}
With the increasing use of the Internet, many businesses and institutions are utilizing online platforms to manage customer inquiries. Consequently, a growing number of them are adopting chatbots to enhance customer service, streamline operations, and boost productivity (Suta et al., 2020). In recent years, chatbots have become an important tool across various industries, particularly in service delivery and automation.

The word “chatbot” is a portmanteau word that is a combination of the words “chatting” and “robot” \cite{rese2020}. A chatbot is an example of technology that is used in computer-mediated communication, where an intelligent system occupies roles once served by humans \cite{beattie2020}. It is also defined as conversational software that is capable of simulating human conversation with an end user through text or voice interaction \cite{Naruzzaman}.

Chatbots can be broadly categorized into two types; rule-based and AI-based chatbots. Rule-based chatbots function with a set of guidelines through pattern-matching and are limited in their conversation. This means that it can only respond to a limited range of queries and vocabulary. AI-based chatbots leverage artificial intelligence(AI), natural language processing(NLP), and machine learning(ML) technologies and algorithms to understand different keywords that users type in when chatting with them. This integration significantly enhances user experience and operational efficiency as these chatbots learn and adapt over time \cite{Kar2016}.

\subsection{Chatbots in Service Automation}
Chatbots are deployed across different platforms, including websites, social media, and instant messaging applications, making them good tools for both internal and external organizational tasks (Hagberg et al., 2016; Zarouali et al., 2018). Internally, chatbots support services, including IT Service Management (ITSM), Human Resource Management (HRM), and learning management systems (Wolf, 2020; Nawaz\& Gomes, 2019; Bakouan, 2018). Externally, chatbots are increasingly replacing traditional branded websites, offering a more interactive platform for customer relationship management, sales, and marketing (Broeck, 2019).

Institutions are utilizing chatbots for various applications. For instance, Pennsylvania State University employs a chatbot called “LionChat” to address frequently asked questions regarding admissions, student aid, and tuition costs \cite{Pennstate}. In healthcare, AI chatbots can be utilized to enhance patient care and streamline processes such as checking symptoms, reminders, and appointment scheduling(Altamimi et al., 2023). Moreover, a case study by \cite{fan2021} on the utilization of a self-diagnosis chatbot in China highlighted the potential for chatbots to improve user engagement by offering real-time feedback and personalized responses.


\section{Synthesis}

As previously mentioned, the researchers aim to create a workflow automation system specifically for the University of the Philippines Visayas Regional Research Center (UPV RRC) to streamline and optimize their service flow and data management. Currently, the institution is using manual processes employing tools such as Google apps.

The difficulty of manual service handling is discussed in this chapter, as well as the benefits of having an automated system. Several studies mentioned indicate that workflow automation can significantly streamline repetitive tasks, improve data accuracy, and enhance decision-making processes reducing human intervention. 

While existing systems for workflow automation are available, there are still gaps that these systems cannot fill, such as limitations with customization, cost-effectiveness, adaptability, and integration issues. The proposed system for UPV RRC aims to address these specific gaps by offering a tailored solution that meets the specific needs of the institution. One technology that can be particularly beneficial for this is a chatbot, which will enhance the consultation process when availing a service from the institution by providing instant responses to inquiries.














