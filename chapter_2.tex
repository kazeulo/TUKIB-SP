%   Filename    : chapter_2.tex 
\chapter{Review of Related Literature}
\label{sec:relatedlit}

The purpose of this literature review is to provide a comprehensive background on automated systems for workflow automation, especially on service processes, which will inform the development of the system for the University of the Philippines Visayas - Regional Research Center (UPV RRC). This review aims to identify existing solutions, highlight gaps and challenges, and explore technologies that can be used to develop the system to improve the UPV RRC’s operational efficiency.

\section{Challenges in Manual Service Handling}

Manual handling of service processes and data management can often lead to challenges, including inefficiencies, errors, and delays. One of the most common issues is the risk of data entry errors. Even small data entry errors can devastate outcomes, corrupting important data. A study involving three different data entry methods (double entry, visual checking, and single entry) revealed that manual entry, particularly visual checking, has a significantly higher number of errors-2958\% more than double entry methods \cite{barchard2011}. These errors can be subtle and difficult to detect, compounding their negative impact on operational efficiency. 

Another limitation of manual service handling is its reliance on human intervention, which frequently results in mistakes that are hard to correct. These errors can escalate operational costs, affect service quality, and lead to customer dissatisfaction. For organizations with manual systems, human error compromises not just data integrity but also the scalability and effectiveness of service operations. Additionally, manual systems lack real-time monitoring capabilities, which are critical for improving service processes. Without automated tracking tools, organizations often miss out on insights that could highlight areas needing improvement.

Current practices in manual service handling also highlight limitations in widely used tools like Google Sheets and Google Docs, which are often insufficient for managing large-scale workflows. These tools lack advanced data retrieval capabilities, and users have reported issues with data not being pulled correctly. According to Okta’s documentation on Google Sheets limitations, there are significant challenges when retrieving and integrating data, leading to inefficiencies in data management processes. Moreover, manual entry in Google Forms is prone to errors, which can undermine the accuracy of collected data.

The impact of these manual methods on stakeholders is substantial. Organizations relying on manual workflows often experience extended processing times, directly affecting service delivery. For instance, tasks that could be automated are unnecessarily prolonged when handled manually, delaying customer satisfaction. Furthermore, manual systems offer limited visibility and tracking capabilities. Without real-time performance metrics, organizations cannot effectively monitor their workflows or identify improvement areas. According to research, companies that automate their workflows experience reduced errors and faster processing times, which lead to improved operational efficiency and better customer outcomes. Thus, the inefficiencies inherent in manual service handling are a barrier to organizational growth, and stakeholders across all levels—from employees to customers—are adversely affected (Davis, n.d.).

\section{Workflow Automation}

Workflow automation refers to the utilization of technology systems, usually involving several software and hardware integrations, to efficiently carry out repetitive tasks, thereby reducing the roles of humans in it \cite{winarko2021}. Workflow automation simplifies the sequencing and completion of tasks within a process by minimizing manual input. Also known as business process automation (BPA), this approach replaces human intervention with digital technologies to automate workflows. At the core of workflow automation is the ability to streamline processes in various job functions—such as HR, accounting, and procurement—into a series of repeated steps without human involvement. Users can define these steps and use tools like drag-and-drop interfaces to create automated workflows. 

Research indicates that automating business processes through workflow automation can re-engineer operations, increase productivity, and improve decision-making timeliness \cite{abecker2000, aversano2002}. It can also enhance efficiency, ensure quality data collection, and improve overall output quality \cite{pakdil2009}. Suitable processes for automation typically exhibit characteristics such as repeatability and predictability \cite{basu2002}. 

A workflow automation software uses rule-based logic to automate tasks that would otherwise require manual effort, such as data entry. While traditionally seen as a tool for IT departments, this software simplifies complex business operations, enhancing efficiency, productivity, and overall customer satisfaction. It is a valuable resource across the entire organization. Connecting various business processes automates critical tasks, sequences, and approvals, allowing workflows to progress automatically without human intervention. This leads to several key advantages for businesses \cite{servicenow}.

Automating workflows offers significant benefits by addressing the limitations and inefficiencies associated with manual processes. While employees are crucial assets, their capacity to handle repetitive tasks is limited, and relying solely on them can lead to bottlenecks, errors, and revenue loss. By automating key steps and handoffs, workflows proceed more swiftly, reducing the time spent on manual tasks and enabling employees to focus on strategic initiatives. Furthermore, automated workflows provide transparency and detailed records, which improve accountability by clearly documenting task progress and responsibilities. Automation also minimizes errors by adhering to predefined rules and methodologies set by programmers, maintaining consistent results.

\subsection{Workflow automation in different industries}

Automation was used for several workflows across a range of industries. Certain industries, like manufacturing and banking, have a long history of using automation, while others, such as legal consultation, hospitality, and transportation, are newer to automation \cite{caban2021}. Across industries, various workflows have been automated, such as accounting tasks, document routing, resource allocation, quality monitoring and control, report generation, and supply chain and logistics management \cite{aguirre2017, mcquilken2014}.

In the education sector, many universities worldwide use automation tools of some form, driven by the need for efficiency and compliance with educational standards. These tools facilitate various processes, including enrollment, grading, and course management, allowing educators to focus more on teaching and student engagement \cite{choudhary2024}. Similarly, automation in healthcare has improved the accuracy and accessibility of patient information, resulting in more informed decision-making. Even in government offices, the evident use of automation tools for service processes can also be observed to enhance service efficiency and transparency. 

\section{Existing Systems}

The development of various digital automation systems and platforms has proliferated over the years. These systems encompass a wide range of functionalities - from automating tasks to facilitating collaboration among staff.

For instance, Enterprise Resource Planning (ERP) Systems are integrated software solutions that manage the core business processes of an organization \cite{blahusiakova2023}. ERP systems integrate various business processes, such as Finance, Human Resources, Supply Chain Management, and Customer Relationship Management (CRM), into one complete system to streamline processes and information across the organization \cite{kimberling2024}. Examples of existing ERP systems that are used by businesses and organizations are Microsoft Dynamics 365 Business Central, Syspro, QT9, and Acumatica. In addition to these comprehensive systems, some businesses and institutions are also utilizing Google apps like Google Drive, Docs, and Sheets to facilitate easier information sharing, enabling teams to work collaboratively. 
 
Moreover, online automation platforms like Zapier and Integromat (Make) help automate interactions between different apps, enabling businesses to integrate multiple systems and optimize workflows without the need for coding (Wolf, 2020). These systems are examples of how institutions tackle complex tasks, reduce manual data entry, and improve decision-making.

\section{Gaps in the  existing systems and solutions}

Despite the availability of various existing automation systems, significant gaps persist that hinder their effectiveness. One major gap is customization limitations which prevent organizations from tailoring solutions to their specific workflows \cite{aleixo2010}. These one-size-fits-all solutions can lead to inefficiencies, as standardized systems may not align with different organizations' unique processes or requirements. Employees might adapt their workflows to fit the software rather than the software, enhancing their operational efficiency. 

Additionally, the lack of adaptability to changing processes can render these existing systems ineffective over time. While these existing solutions might be beneficial to some companies, they can be detrimental to organizations that rely on their capacity to meet customer demands\cite{akkermans2003}. Also, as organizations evolve, they often need to adjust their workflows in response to new challenges, regulations, or market demands. Rigid Systems that cannot easily accommodate such changes can become obsolete. 

Furthermore, many existing software solutions are proprietary, increasing costs for organizations. Proprietary systems often have high licensing fees, maintenance costs, and limited scalability \cite{goel2012, prasad2013}. Organizations may find themselves locked into contracts that are not cost-effective, particularly if the software does not deliver the expected return on investment. On top of that, the difficulty of adapting and getting these automation systems to work effectively is also well documented \cite{adams2011, sarker2003, scott2000}.


\section{Chatbot}

With the increasing use of the Internet, many businesses and institutions are utilizing online platforms to manage customer inquiries. Consequently, a growing number of them are adopting chatbots to enhance customer service, streamline operations, and boost productivity (Suta et al., 2020). In recent years, chatbots have become an important tool across various industries, particularly in service delivery and automation. Inarguably, chatbots are used daily by some people. Some instances of this are Siri from Apple, Alexa from Amazon, Microsoft Cortana and Bixby from Samsung that have the ability to open apps, play music, set calendar events and, overall, be a virtual assistant.

The word “chatbot” is a portmanteau word that is a combination of the words “chatting” and “robot” \cite{rese2020}. A chatbot is an example of technology that is used in computer-mediated communication, where an intelligent system occupies roles once served by humans \cite{beattie2020}. It is also defined as conversational software that is capable of simulating human conversation with an end user through text or voice interaction \cite{Naruzzaman}.

Chatbots can be broadly categorized into two types; rule-based and AI-based chatbots. Rule-based chatbots function with a set of guidelines through pattern-matching and are limited in their conversation. This means that it can only respond to a limited range of queries and vocabulary. AI-based chatbots leverage artificial intelligence(AI), natural language processing(NLP), and machine learning(ML) technologies and algorithms to understand different keywords that users type in when chatting with them. This integration significantly enhances user experience and operational efficiency as these chatbots learn and adapt over time \cite{Kar2016}.

\subsection{Chatbots in Service Automation}
Chatbots are deployed across different platforms, including websites, social media, and instant messaging applications, making them good tools for both internal and external organizational tasks \cite{hagberg2016, zarouali2018}. Internally, chatbots support services, including IT Service Management (ITSM), Human Resource Management (HRM), and learning management systems \cite{nawaz2019, bakouan2018}. Externally, chatbots are increasingly replacing traditional branded websites, offering a more interactive platform for customer relationship management, sales, and marketing \cite{broeck2019}.

Institutions are utilizing chatbots for various applications. For instance, Pennsylvania State University employs a chatbot called “LionChat” to address frequently asked questions regarding admissions, student aid, and tuition costs \cite{Pennstate}. In healthcare, AI chatbots can be utilized to enhance patient care and streamline processes such as checking symptoms, reminders, and appointment scheduling \cite{altamimi2023}. Moreover, a case study by \cite{fan2021} on the utilization of a self-diagnosis chatbot in China highlighted the potential for chatbots to improve user engagement by offering real-time feedback and personalized responses.

\subsection{Chatbot Frameworks}

Building a chatbot from scratch is not an easy task, that’s why chatbot development frameworks have emerged over the years. Chatbot development frameworks are software frameworks that provide built-in functions that simplify the complexities of creating a chatbot \cite{geekforgeeks2024}.  A study by Zahour et. al, 2020, provides a comprehensive analysis of various frameworks that can be used in developing chatbots. \ref{Tab: framework_comparison} shows the analysis between different frameworks white table 2 shows the difference between AI chatbots and non-AI chatbots.

\begin{table}[]
	\begin{adjustbox}{max width=\textwidth}
		\begin{tabular}{|l|l|l|l|l|l|l|l|l|l|}
		\hline
		\textbf{Framework} 
		& \textbf{Company} 
		& \textbf{Paid/Free} 
		& \textbf{Ease of Use} 
		& \textbf{Out of box integration} 
		& \textbf{Open Source} 
		& \textbf{Popularity} 
		& \textbf{Web-based} 
		& \textbf{Language} 
		\\ \hline
		\textbf{QnA Maker} 
		& Microsoft 
		& Free 
		& High 
		& Yes 
		& No 
		& Medium 
		& Yes 
		& C\# 
		\\ \hline
		\textbf{DialogFlow} 
		& Google 
		& Free 
		& High 
		& Yes 
		& No 
		& High 
		& Yes 
		& JavaScript 
		\\ \hline
		\textbf{RASA} 
		& RASA 
		& Free 
		& Low 
		& No 
		& Yes 
		& High 
		& No 
		& Python 
		\\ \hline
		\textbf{Wit.ai} 
		& Facebook 
		& Free 
		& High 
		& Yes (Facebook) 
		& No 
		& High 
		& Yes 
		& JavaScript 
		\\
		\hline
		\textbf{Luis.ai} 
		& Microsoft 
		& Free 
		& High 
		& Yes 
		& No 
		& Medium 
		& Yes 
		& JavaScript 
		\\
		\hline
		\textbf{Botkit.ai} 
		& Botkit 
		& Free 
		& Low 
		& Yes 
		& No 
		& Medium 
		& No 
		& JavaScript 
		\\
		\hline
		\end{tabular}
	\end{adjustbox}
	\caption{Analysis of different chatbot development frameworks.}
	\label{Tab: framework_comparison}
\end{table}

\begin{table}
	
\end{table}

\section{Synthesis}

As previously mentioned, the researchers aim to create a workflow automation system specifically for the University of the Philippines Visayas Regional Research Center (UPV RRC) to streamline and optimize their service flow and data management. Currently, the institution is using manual processes employing tools such as Google apps.

The difficulty of manual service handling is discussed in this chapter, as well as the benefits of having an automated system. Several studies mentioned indicate that workflow automation significantly enhance operational efficiency by reducing repetitive tasks, improving data accuracy, and making data management easier.

While many systems for workflow automation are available, there are still gaps that these systems cannot fill, such as limitations with customization, cost-effectiveness, adaptability, and integration issues. The proposed system for UPV RRC aims to address these specific gaps by offering a tailored solution that meets the specific needs of the institution. One technology that can be particularly beneficial for this is a chatbot, which will enhance the consultation process when availing a service from the institution by providing around-the-clock support and instant responses to inquiries. This benefits not only the staff of the UPV RRC but the clients as well.












