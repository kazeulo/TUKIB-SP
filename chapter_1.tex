%   Filename    : chapter_1.tex 
\chapter{Introduction}
\label{sec:researchdesc}    %labels help you reference sections of your document

\section{Overview of the Current State of Technology}
\label{sec:overview}

In the era of digital transformation, efficient data management and streamlined service workflows are critical for the success of any business or institution. Perhaps one of the remarkable and known products of technology is converting paper-based or manually-operated systems to automated systems. It is irrefutable that automation greatly impacts people's lives, providing increased efficiency and productivity.

The University of the Philippines Visayas - Regional Research Center (UPV RRC) is a centralized facility that strengthens UP Visayas’ research and innovation capabilities by providing researchers access to and training on advanced analytical equipment and method development. It provides several services catering to different fields of natural and physical sciences. Current practices on the service flow of the institution rely heavily on manual processes, using tools such as Google Forms and Google Sheets for service request handling, tracking, and data management. While these methods provide a foundational level of functionality, they fall short in addressing the specific needs of service flow requirements of the RRC, posing challenges not only for the staff but also for the clients. The need for a more sophisticated and integrated system led to the conceptualization of TUKIB.

The proliferation of advanced software solutions presents an opportunity to enhance operational efficiency by automating service flow tasks. However, existing systems fail to provide the specific necessities of some institutions hence, a more sophisticated software is often needed. By developing a specialized software solution tailored to the unique needs of RRC, it is possible to significantly improve productivity, data accuracy, and overall effectiveness. This project explores the design and implementation of such a software solution, aiming to replace the existing reliance on Google Apps with a more robust, integrated system.
  
The proposed software seeks to address several key challenges faced by the institution, including the automation of repetitive tasks and the facilitation of seamless communication among team members. By leveraging modern technologies and best practices in software development, this research aims to provide a practical, scalable solution that can be adapted to various research environments.

\section{Problem Statement}


The UPV Regional Research Center (RRC) currently relies on a manual service workflow for handling client requests, managing laboratory services, and tracking research-related activities. This process, which depends on Google Forms and Sheets, lacks automation, leading to inefficiencies such as delays in service requests, difficulty in tracking progress, and limited scalability as the demand for RRC services grows. Furthermore, the absence of a centralized system makes it challenging for staff to manage and monitor multiple services and for clients to access real-time information about their requests.

To address these issues, a comprehensive and integrated workflow automation system, named TUKIB, is necessary. The system aims to automate service requests, improve data management, enhance communication between RRC staff and clients, and streamline overall operations. With automation, the center can improve the efficiency, accuracy, and accessibility of its services, supporting both the internal management and external user experience.

\section{Research Objectives}
\label{sec:researchobjectives}

\subsection{General Objective}
\label{sec:generalobjective}

The general objective of this paper is to develop a system to automate and optimize the service flow and data management at UPV Regional Research Center and evaluate its effectiveness. The system will be called TUKIB, an acronym for Tracking Utility for Knowledge Integration and Benchmarking. 


\subsection{Specific Objectives}
\label{sec:specificobjectives}

\begin{comment}
% IPR acknowledgement: the following sentences and examples are from Ethel Ong's slides 
%     on Research Objectives
How to formulate your research objectives:
1. Identify what research steps do you need to perform to achieve your general objective.
2. Identify the questions that must be answered for you to achieve your general objective.
    Thereafter, convert these questions into action statements

Example #1:

Research Question:
  What are the general features of a web-based learning environment?

Specific Objective:
   To review existing web-based learning environment that teaches language learning for children


Example #2:

Research Question:
   How will you represent commonsense knowledge for use by computer systems?

Specific Objective:
   To identify knowledge representation approaches used by existing story generation systems

Example #3:
Research Question:
   What types of storytelling knowledge are needed to generate stories?

Specific Objective:
    To identify the different types of storytelling knowledge used in generating stories

Example #4:
Research Question:
    What machine learning approaches will you utilize?

Specific Objective:
    To determine existing machine learning algorithms [that can be used in training the computer system to detect cyberbullying cases] 

Example #5: Research Question:
    How will your research output be evaluated?

Specific Objective:
    To define evaluation metrics for validating the accuracy of the translation

\end{comment}

Specifically this study aims to:

\begin{enumerate}
   \item Automate the management of service requests and tracking, enabling real-time monitoring of ongoing tasks and requests for both RRC staff and clients by developing an integrated workflow automation system that streamlines the UPV Regional Research Center's (RRC) service processes, reducing manual intervention and enhancing operational efficiency.
   
   \item Create a centralized data management system for RRC that ensures secure, efficient storage and retrieval of information related to service requests, laboratory usage, and client transactions.
   
   \item Improve communication and feedback mechanisms between RRC staff and clients, enabling the RRC to get the necessary statistics for customer satisfaction, and identify their service strengths and weaknesses.
   
   \item Design and implement a chatbot, allowing the automation of the initial consultation process and for clients to interact with the system for service inquiries and assistance, providing immediate and accurate responses.
   
   \item Evaluate the system’s impact on operational efficiency, compare the automated workflow with the previous manual processes in terms of speed, accuracy, and user satisfaction.
   
   \item Ensure the system is scalable and adaptable to future requirements, allowing the RRC to accommodate increased demand and potentially integrate additional features in the long term.
   
\end{enumerate}


\section{Scope and Limitations of the Research}
\label{sec:scopelimitations}

This special problem focuses on developing the Tracking Utility for Knowledge Integration and Benchmarking (TUKIB), an integrated workflow automation system designed for the UPV Regional Research Center (RRC). The system aims to automate key service flow and data management aspects within the RRC. 

TUKIB will cover the full-service management cycle within the UPV RRC, from initial client service requests to the completion and feedback stage. It will include features such as real-time tracking of service requests, a full inventory list and management of the RRC equipment, automated notifications to clients and staff, and an integrated platform for storing and managing service data. Key components such as user interfaces for staff and clients, real-time service status updates, events and schedule management, transaction records,  and a feedback collection mechanism will be developed. Data accuracy will be ensured by minimizing manual input and automating repetitive processes, reducing errors and improving operational efficiency. The project will also involve the deployment of chatbots to enhance the communication flow between clients and staff, providing instant responses to inquiries and updates on service requests. The system will be scalable, allowing it to be adapted to other similar research institutions in the future.

The system’s functionalities will be limited to the services provided by the UPV RRC and may not cover other external functions or services. Customization will be tailored to the specific workflows of UPV RRC, so further modification would be needed for implementation in different institutions or industries. The project will focus on workflow automation but will not delve into advanced analytics or AI beyond using chatbots for customer communication and statistics for service feedback reports. The system requires a stable internet connection for real-time features like notifications and status tracking; thus, its performance may be compromised in areas with poor connectivity. The effectiveness of the system depends on staff and client adaptability to the new system, which may require a period of training and adjustment.


\begin{comment}

%
% IPR acknowledgement: the sentences inside this comment are from Ethel Ong's slides on Scope and Limitations of the Research
%
Generally, one paragraph should be allotted for each of your research objectives.

Each paragraph contains a brief overview of the concept/theory and the purpose of doing the associated objective.

Each paragraph also includes a description of the scope/limitation of your study.

* Please refer to the slides for examples.

\end{comment}


\section{Significance of the Research}
\label{sec:significance}

The development of TUKIB offers significant contributions on multiple fronts, benefiting the researchers, the UPV RRC, and other research institutions facing similar challenges in service and data management, the computer science community, and the general society.

\begin{itemize}
	
\item \textbf{The Researchers}

\subitem The TUKIB project provides an invaluable opportunity for researchers to apply their theoretical knowledge and practical skills to solve real-world problems. It allows them to demonstrate their competency in system design, workflow automation, and software development, contributing to the completion of their degree requirements.
	
\subitem Beyond academic fulfillment, the project also equips the researchers with hands-on experience in managing complex systems, collaborating with stakeholders, and implementing scalable technological solutions, which will be beneficial in their future careers in computer science and related fields.\newline

\item \textbf{The UPV RRC and Other Research Institutions}

\subitem The TUKIB system will significantly improve the operational efficiency of the UPV Regional Research Center by automating its service request workflows and data management processes. The integration of this system will reduce the time and effort spent on manual tasks such as request processing, service tracking, and data entry. This not only streamlines the internal processes but also enhances the overall user experience for both researchers and external clients, who will benefit from a more transparent and efficient service flow.

\subitem Furthermore, other research institutions facing similar challenges in managing their services and data will be able to adapt TUKIB to their own workflows, allowing them to optimize resource allocation and improve communication between staff and clients. TUKIB’s customizable and scalable nature makes it a valuable model for research institutions looking to enhance their operations without investing in entirely new systems.\newline

\item \textbf{The Computer Science Community}

\subitem For the computer science community, TUKIB represents a meaningful contribution in terms of integrating workflow automation, real-time tracking, and chatbot technology into a research-driven service environment. The project showcases an innovative approach to solving a niche problem, providing a practical application for the latest software development methods and techniques in workflow optimization. Additionally, it demonstrates the importance of developing scalable, customizable solutions that can be adapted to a variety of organizational contexts.

\subitem This research also serves as a case study in designing user-centered automation systems, contributing to the knowledge of software solutions that bridge the gap between operational requirements and technological advancements. The learnings from TUKIB could inspire future research in workflow management, data accuracy, and intelligent user interfaces.\newline

\item \textbf{The General Society}

\subitem On a broader scale, the TUKIB project has the potential to benefit society by promoting more efficient research processes. By optimizing how research institutions manage their services, TUKIB indirectly supports the advancement of scientific research. With more streamlined workflows and reduced administrative burdens, research institutions can focus their resources on the core activities of scientific discovery and innovation. This, in turn, may lead to faster advancements in areas like environmental science, technology development, and public health, which could have far-reaching societal impacts.\newline

\subitem In summary, TUKIB stands as an important system not only for those immediately involved in its implementation but also for the larger community of researchers, developers, and society as a whole. Its contributions reach across the fields of computer science, research, and institutional management, offering lasting benefits in terms of technological innovation and service improvement.

\end {itemize}

