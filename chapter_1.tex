%   Filename    : chapter_1.tex 
\chapter{Introduction}
\label{sec:researchdesc}    %labels help you reference sections of your document

\section{Overview}
\label{sec:overview}

In the era of digital transformation, efficient data management and optimized service workflows are crucial for the success of any business or institution. Perhaps one of the most remarkable and well-known products of technology is the conversion of paper-based or manually-operated systems to automated systems. It is unquestionable that automation greatly impacts people's lives, providing increased efficiency and productivity.

The University of the Philippines Visayas - Regional Research Center (UPV RRC) is a centralized facility that strengthened UP Visayas’ research and innovation capabilities by providing researchers access to and training on advanced analytical equipment and method development. It provides several services catering to different fields of natural and physical sciences. At the time, the institution relies heavily on manual processes, using tools such as Google Apps throughout its entire service delivery process, from handling service requests and tracking to data management. Although this method offers a foundational level of functionality, it fails to address the specific needs of the UPV RRC in its service delivery workflow. This poses challenges not only for the staff but also for the clients of the institution.

Automation, defined as “the application of technology, programs, robotics or processes to achieve outcomes with minimal human input” \cite{ibm}, has been effectively adopted across various industries to enhance quality, productivity, efficiency, timeliness, effectiveness, and operational safety. It also helps in reducing costs and provides greater value to customers \cite{caban2021}. 

Over the years, various technologies emerged to address the pressing need for automation. The increase in advanced software solutions offered organizations and institutions an opportunity to enhance their operational efficiency. However, existing systems fell short in addressing the specific needs of some institutions. Adapting these existing systems often guve birth to other problems as integrating and customizing off-the-shelf softwares can be difficult, costly, and limited in scalability \cite{bitcat2023}. In such cases, developing new software tailored to the specific needs of an institution is often a better option.

Recognizing this gap, this study explored the design and implementation of a software solution tailored to the unique needs of the UPV RRC, aiming to replace the institution’s current system by automating its service delivery flow and data management. Additionally, this paper included the development and integration of a chatbot using the Rasa framework to enhance and streamline the institution’s client support, interaction, and communication. By using modern technologies and best practices in software development, this study sought to add knowledge on building a practical and scalable system, specifically one that could be used by the UPV RRC for their service delivery processes.

\section{Problem Statement}

In today’s fast-paced world, success is often associated with efficiency, especially in the business environment. A report by McKinsey \& Company \cite{manyika2017} reveals that about 60\% of occupations involve at least 30\% of tasks that were automatable. Despite the growing recognition for the need of automation and even with the rise of different technologies, many businesses and institutions are still dependent on manual or semi-automated workflows. While various workflow automation technologies exist, adapting off-the-shelf softwares is costly and challenging as these softwares often requires extensive customization to fit the institution’s unique needs and are difficult to integrate. 

The University of the Philippines Visayas Regional Research Center (UPV RRC) is one such institution that is still reliant on manual processes, especially on its service flow delivery and data management. Various tasks including handling of client requests, managing laboratory services, and tracking service-related activities are carried out with the use of semi-automated tools like Google Apps, which is technically still dependent on human intervention. This leads to inefficiencies such as delays, difficulty in tracking, and vulnerability to errors, ultimately compromising the institution’s overall productivity.

To address these issues, an integrated workflow automation system tailored to the needs of the UPV RRC was developed to ease the difficulties faced by the institution in its service delivery. This system automated service requests, streamlined data management, enhanced communication between RRC staff and clients, and improved overall operations. With automation, the center improved the efficiency, accuracy, and accessibility of its services, supporting both internal management and the external client experience.

\section{Research Objectives}
\label{sec:researchobjectives}

\subsection{General Objective}
\label{sec:generalobjective}

The general objective of this paper is to develop a system to automate and optimize the service flow and data management at UPV Regional Research Center and evaluate its effectiveness. The system will be called TUKIB, an acronym for Tracking Utility for Knowledge Integration and Benchmarking. 

\subsection{Specific Objectives}
\label{sec:specificobjectives}

Specifically this study aims to:

\begin{enumerate}
   
   \item develop a centralized data management system for the RRC to ensure secure, efficient storage, retrieval, and management of information related to service requests, laboratory usage, and client transactions,
   
   \item design and implement an automated chatbot to handle consultations and frequently asked questions, enabling clients to interact with the system for service inquiries and support in real-time,
   
   
   \item implement an intuitive and user-friendly design that ensures ease of use and accessibility for both staff and clients of UPV RRC, and
   
   \item evaluate the system’s impact on operational efficiency, and compare the automated workflow with the previous manual processes in terms of speed, accuracy, and user satisfaction.
   
\end{enumerate}


\section{Scope and Limitations of the Research}
\label{sec:scopelimitations}

As mentioned, this special problem focuses on developing TUKIB - short for Tracking Utility for Knowledge Integration and Benchmarking, a workflow automation system designed for the UPV Regional Research Center (RRC). 

TUKIB covers the full-service management cycle of the UPV RRC, from initial client service requests to the completion and service feedback stage. It has features such as real-time tracking of service requests, facility and equipment availability tracking, and a centralized platform for storing and managing service-related data. Key components such as user interfaces for staff and clients, real-time service status updates, events, service schedule management, transaction records, and a feedback collection mechanism were also added. With this, data accuracy throughout the service flow process is ensured by minimizing manual input and automating repetitive processes, reducing errors and improving the operational efficiency of the UPV RRC. This special problem also involved the development and integration of a chatbot to enhance user support and communication between clients and staff, providing instant responses to inquiries. Additionally, the system was made scalable, allowing it to be flexible for further modification as the needs of UPV RRC evolved.

The system’s functionalities were limited to the service-related processes of the UPV RRC and did not cover other internal and external functions. The development was tailored to the specific workflows of UPV RRC, so modifications would be needed for implementation in different institutions or industries. Additionally, this special problem focused on workflow automation but did not delve into advanced analytics or AI beyond using chatbots for customer communication and basic statistics for service feedback reports. The system required a stable internet connection for real-time features like notifications and status tracking; thus, its performance could be compromised in areas with poor connectivity. Moreover, the effectiveness of the system depended on staff and client adaptability to the new system, which required a period of training and adjustment.

\section{Significance of the Research}
\label{sec:significance}

This study offers great significance in many domains, benefiting the UPV RRC and its clients, the researchers, other institutions, the computer science community, and future researchers.

\begin{itemize}
	
\item \textbf{The Researchers}

This study provides a great opportunity for the researchers to apply their theoretical knowledge and practical skills to solve real-world problems. This allows them to demonstrate their competency in system design and software development. \newline

\item \textbf{The UPV RRC and its Clients}

The development of TUKIB will significantly improve the operational efficiency of the UPV Regional Research Center by automating its service request workflows and data management processes. This will not only benefit the staff but the clients as well.\newline

\item \textbf{Other Institutions}

Other institutions facing similar challenges in managing their service flow processes and data can also benefit from this special problem. They can adapt TUKIB to their own workflows or this study can serve as a guide for them in creating their own specialized software solution. \newline

\item \textbf{The Computer Science Community}

The Computer Science Community also benefits from this study. This paper contributes to the existing knowledge in developing a tailored workflow automation system by providing perspective into the practical application of various software development tools and  methods. Additionally, this special problem also serves as a case study in designing a user-centered software. Other developers can gain valuable insights and inspiration from this for their own projects.\newline

\item \textbf{Future Researchers}

The special problem can serve as a reference and guide for future
researchers who wish to pursue studies similar or related to this special problem.\newline

\end {itemize}

