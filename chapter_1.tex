%   Filename    : chapter_1.tex 
\chapter{Introduction}
\label{sec:researchdesc}    %labels help you reference sections of your document

\section{Overview}
\label{sec:overview}

In the era of digital transformation, efficient data management and streamlined service workflows are critical for the success of any business or institution. Perhaps one of the most remarkable and known products of technology is converting paper-based or manually-operated systems to automated systems. It is irrefutable that automation greatly impacts people's lives, providing increased efficiency and productivity.

The University of the Philippines Visayas - Regional Research Center (UPV RRC) is a centralized facility that strengthens UP Visayas’ research and innovation capabilities by providing researchers access to and training on advanced analytical equipment and method development. It provides several services catering to different fields of natural and physical sciences. Current practices on the service flow of the institution rely heavily on manual processes, using tools such as Google Apps for their entire service delivery process- from request handling, tracking, to data management. Although this method is functional, it falls short in addressing the specific needs of service flow requirements of the RRC, posing challenges not only for the staff but for the clients as well. 

Automation, defined as “the application of technology, programs, robotics or processes to achieve outcomes with minimal human input” \cite{ibm}, has been effectively adopted across various industries to enhance quality, productivity, efficiency, timeliness, effectiveness, and operational safety. It also helps in reducing costs and provides greater value to customers \cite{caban2021}. Over the years, various technologies have emerged to address the pressing need for automation. The increase in advanced software solutions presents an opportunity to enhance operational efficiency by automating certain tasks and processes. However, existing systems fail to provide the specific necessities of some institutions hence, a more specialized software is often needed. 

By developing a specialized software solution tailored to the unique needs of RRC, it is possible to significantly improve productivity, data accuracy, and overall efficiency of the institution. This study explores the design and implementation of such a software solution. The primary purpose of the system is to automate the service flow and data management of the UPV RRC, aiming to replace the institution's current system and minimize reliance on Google Apps. Additionally, this study also includes the development and integration of a chatbot to enhance user support and communication with stakeholders.

The proposed software aims to address several key challenges faced by the UPV RCC, including the automation of repetitive tasks, enhanced data management, and smoother communication among stakeholders. With the use of modern technologies and best practices in software development, this research seeks to provide a practical, scalable solution that can be used by the UPV RRC and adapted by other institutions with similar operations and needs.

\section{Problem Statement}

The UPV Regional Research Center (RRC) currently relies on a manual service workflow for handling client requests, managing data, and tracking service-related activities. These processes, which are dependent on Google Forms and Sheets are difficult to organize and prone to error, leading to inefficiencies such as delays in service requests, difficulty in tracking progress, and limited scalability as the demand for RRC services grows. Furthermore, the absence of a centralized system makes it challenging for staff to manage and monitor multiple services and for clients to access real-time information about their requests.

To address these issues, a comprehensive and integrated workflow automation system is necessary. The system aims to automate service requests, improve data management, enhance communication between RRC staff and clients, and streamline overall operations. With automation, the center can improve the efficiency, accuracy, and accessibility of its services, supporting both the internal management and external customer experience.

\section{Research Objectives}
\label{sec:researchobjectives}

\subsection{General Objective}
\label{sec:generalobjective}

The general objective of this paper is to develop a system to automate and optimize the service flow and data management at UPV Regional Research Center and evaluate its effectiveness. The system will be called TUKIB, an acronym for Tracking Utility for Knowledge Integration and Benchmarking. 

\subsection{Specific Objectives}
\label{sec:specificobjectives}

Specifically this study aims to:

\begin{enumerate}
   
   \item Create a centralized data management system for RRC that ensures secure, efficient storage and retrieval of information related to service requests, laboratory usage, and client transactions.
   
   \item Design and implement a chatbot, allowing the automation of the consultation process and for clients to interact with the system for service inquiries and assistance, providing immediate and accurate responses.
   
   \item Evaluate the system’s impact on operational efficiency, compare the automated workflow with the previous manual processes in terms of speed, accuracy, and user satisfaction.
   
\end{enumerate}


\section{Scope and Limitations of the Research}
\label{sec:scopelimitations}

This special problem focuses on developing the TUKIB- short for Tracking Utility for Knowledge Integration and Benchmarking, a workflow automation system designed for the UPV Regional Research Center (RRC). 

TUKIB will cover the full-service management cycle of the UPV RRC, from initial client service requests to the completion and feedback stage. It will include features such as real-time tracking of service requests, facility and equipment availability tracking, and a centralized platform for storing and managing service-related data. Key components such as user interfaces for staff and clients, real-time service status updates, events and schedule management, transaction records, and a feedback collection mechanism will be included in the development. With this, data accuracy throughout the service flow process will be ensured by minimizing manual input and automating repetitive processes, reducing errors and improving operational efficiency of the UPV RRC. This special problem will also involve the development and integration of a chatbot to enhance user support and communication between clients and staff, providing instant responses to inquiries and updates on service requests. Additionally, the system will be scalable, allowing it to be flexible for further modification and be adapted by other institutions with similar needs.

The system’s functionalities will be limited to the service-related processes by the UPV RRC and may not cover other internal and external processes and functions of the institution. The development will be tailored to the specific workflows of UPV RRC, so modifications would be needed for implementation in other institutions or industries. Additionally, this special problem will focus on workflow automation but will not delve into advanced analytics or AI beyond using chatbots for customer communication and basic statistics for service feedback reports. The system will require a stable internet connection for real-time features like notifications and status tracking; thus, its performance may be compromised in areas with poor connectivity. Moreover, the effectiveness of the system depends on staff and client adaptability to the new system, which may require a period of training and adjustment.

\section{Significance of the Research}
\label{sec:significance}

This development of TUKIB offers significant contributions in many domains, benefiting the researchers, the UPV RRC staff and clients, other research institutions that are facing similar challenges, the computer science community, and the general society.

\begin{itemize}
	
\item \textbf{The Researchers}

\subitem The TUKIB project provides a great opportunity for the researchers to apply their theoretical knowledge and practical skills to solve real-world problems. It allows them to demonstrate their competency in system design, workflow automation, and software development, contributing to the completion of their degree requirements.

\subitem Beyond academic fulfillment, the project also equips them with hands-on experience in managing complex systems, collaborating with stakeholders, and implementing a comprehensive system, which will be beneficial in their future careers in computer science and related fields.\newline

\item \textbf{The UPV RRC and its clients}

\subitem The UPV RRC will greatly benefit from this special problem. The development of TUKIB will significantly improve the institution's operational efficiency by automating its service request workflows and data management processes. This will not only benefit the staff of the UPV RRC but their clients as well.\newline

\item \textbf{Other Institutions}

\subitem Other institutions facing similar challenges in managing their service flow processes and data can also benefit from this special problem. They can adapt TUKIB to their own workflows or this can serve as a guide for them in creating their own specialized software solution. TUKIB’s customizable and scalable nature makes it a valuable model for institutions looking to enhance their operations without investing in entirely new systems.\newline

\item \textbf{The Computer Science Community}

\subitem The computer science community also benefits from this special problem. TUKIB has features that showcases an innovative approach to solving a niche problem, providing a practical application for the many software development tools and methods. Additionally, it demonstrates the importance of developing scalable, customizable solutions that can be adapted and tailored to a different and evolving of user needs.

\subitem This special problem also serves as a case study in designing user-centered automation systems. Other developers can gain valuable insights and inspiration from this for their own projects.

\end {itemize}

