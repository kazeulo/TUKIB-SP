%   Filename    : chapter_2.tex 
\chapter{Review of Related Literature}
\label{sec:relatedlit}

This chapter discusses the features, capabilities, and limitations of existing research, algorithms, or software  that are related/similar to the Special Problem.

 The reviewed works and software must be arranged either in chronological order, or by area (from general to specific).  
Observe a consistent format when presenting each of the reviewed works. 
This must be selected in consultation with the  adviser.

\textcolor{red}{DO NOT FORGET to cite your references.}

A literature review must do these things:
\begin{itemize}
	\item be organized around and related directly to the thesis or research question you are developing
	\item synthesize results into a summary of what is and is not known
	\item identify areas of controversy in the literature
	\item formulate questions that need further research
\end{itemize}

A literature review is a piece of discursive prose, not a list describing or summarizing one piece of literature after another. It’s usually a bad sign to see every paragraph beginning with the name of a researcher. Instead, organize the literature review into sections that present themes or identify trends, including relevant theory. You are not trying to list all the materials published, but to synthesize and evaluate them according to the guiding concept of your thesis or research question. You should also state the limits or gaps of their researches wherein you will try to fill these gaps in accordance to your research problem and objectives.
\begin{comment}
%
% IPR acknowledgement: the contents withis this comment are from Ethel Ong's slides on RRL.
%
Guide on Writing your RRL chapter
 
1. Identify the keywords with respect to your research
      One keyword = One document section
                Examples: 2.1 Story Generation Systems
			 2.2 Knowledge Representation

2.  Find references using these keywords

3.  For each of the references that you find,
        Check: Is it relevant to your research?
        Use their references to find more relevant works.

4. Identify a set of criteria for comparison.
       It will serve as a guide to help you focus on what to look for

5. Write a summary focusing on -
       What: A short description of the work
       How: A summary of the approach it utilized
       Findings: If applicable, provide the results
        Why: Relevance to your work

6. At the end of each section,  show a Table of Comparison of the related works 
   and your proposed project/system

\end{comment}

\section{Theme 1 Title}
This chapter  contains a review of research papers that:
%
% IPR acknowledgement: the following list of items are from Ethel Ong's slides RRL.
%
\begin{itemize}
\item Describes work on a research area that is similar or relevant to yours
\item Describes work on a domain that is similar or relevant to yours
\item Uses an algorithm that may be useful to your work
\item Uses a software / tool that may be useful to your work
\end{itemize}

%\section{Review of Related Software}
It also contains a review of software systems that:
%
% IPR acknowledgement: the following list of items are from Ethel Ong's slides on RRL.
%
\begin{itemize}
   \item Belongs to a research area similar to yours
   \item Addresses a need or domain similar to yours
   \item Is your predecessor
\end{itemize}

\section{Theme 2 Title}



\section{Chapter Summary}
Should include a table of related studies comparing them based on several criteria.

Highlight research gaps and the research problem.













