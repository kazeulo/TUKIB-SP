%   Filename    : abstract.tex 
\begin{center}
\textbf{Abstract}
\end{center}
\setlength{\parindent}{0pt}
Manual service flow and data management remain two of the most common challenges faced by many businesses and institutions, even in today’s digital age. One such institution is the University of the Philippines Visayas – Regional Research Center (RRC), which relies on manual and semi-automated processes using Google Apps throughout its service delivery. While functional, this system is inefficient and limits the RRC’s potential, creating challenges for both staff and clients. This study aims to develop a centralized system, aptly named TUKIB, to automate the service flow and data management processes of the UPV RRC. It also explores the development and integration of a chatbot using the Rasa framework. The project adopted an Agile methodology, emphasizing iterative development and regular feedback to ensure the system addressed evolving user needs. The resulting system can significantly reduce manual tasks, improve data management, enhance client support, and streamline operations, which can ultimately increase the overall efficiency of the UPV RRC.

\begin{tabular}{lp{4.25in}}
\hspace{-0.5em}\textbf{Keywords:}\hspace{0.25em} & workflow automation, chatbot, rasa, data management, service flow\\
\end{tabular}