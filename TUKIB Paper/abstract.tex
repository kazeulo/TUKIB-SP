%   Filename    : abstract.tex 
\begin{center}
\textbf{Abstract}
\end{center}
\setlength{\parindent}{0pt}
Manual service flow and data management remain two of the most common challenges faced by many businesses and institutions, even in today’s digital age. One such institution is the University of the Philippines Visayas – Regional Research Center (RRC), which relies on manual and semi-automated processes using Google Apps throughout its service delivery. While functional, this system is inefficient and limits the RRC’s potential, creating challenges for both staff and clients.

The main objective of this study is to develop a centralized system, aptly named TUKIB, to automate the service flow and data management processes of the University of the Philippines Visayas - Regional Research Center. In addition, this study involved the development and integration of a chatbot using the Rasa framework to support initial client interaction and consultation. The project utilized an Agile methodology, focusing on iterative development and regular feedback loops to ensure the system met the evolving needs of the users. 

The results suggest that the developed system has the potential to reduce manual tasks, improve data management, enhance client support, and provide ease for both staff and clients, ultimately enhancing the overall operational efficiency of UPV RRC.

\begin{tabular}{lp{4.25in}}
\hspace{-0.5em}\textbf{Keywords:}\hspace{0.25em} & workflow automation, chatbot, rasa, data management, service flow\\
\end{tabular}