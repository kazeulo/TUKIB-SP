%   Filename    : chapter_5.tex 
\chapter{Conclusion and Recommendations}
This chapter presents the conclusion of the study and offers practical recommendations for future enhancement and implementation of TUKIB.

\section{Conclusion}

With rapid technological advancements, adaptability and flexibility— especially within businesses and institutions— is a must. Failure to keep up with the advancements can render an institution inefficient and potentially lead to its operational failure.

To address this problem, the researchers developed TUKIB— a system designed to automate and streamline service delivery at the UPV Regional Research Center (RRC), an institution that faces such challenges. The system was built with a strong focus on understanding and fulfilling user needs and requirements. The developers also focused on creating an intuitive and user friendly interface to ensure that the service delivery of RRC would be efficient, not only for the staff but for the clients as well. 

TUKIB was developed using HTML5, Bootstrap, and JavaScript for the frontend, while the backend was powered by PostgreSQL and ReactJS. The chatbot component, implemented using the Rasa framework, provided a conversational interface to assist users throughout the platform.

All system interfaces and features were deliberately planned, starting from the data gathering and requirements identification phases through to system design. These activities guided the development process and features were refined incrementally through iterative testing and continuous feedback from stakeholders.

From the development, testing, system assessment, and stakeholder feedback, several key findings were drawn. First, transitioning to an automated, centralized platform like TUKIB is essential for modernizing service management at the RRC, significantly enhancing both efficiency and accuracy over traditional manual processes. Second, the active involvement of stakeholders throughout the design and development process was crucial in ensuring that the system addressed the specific needs and workflows of both staff and clients. Third, the successful implementation of TUKIB offers a valuable model for other research institutions seeking to digitize and optimize their service delivery operations.

Furthermore, the project highlighted the importance of maintaining open and continuous communication between developers and stakeholders. Regular feedback sessions proved vital in aligning system features with evolving user needs and expectations.

While TUKIB addresses several key inefficiencies in service delivery, it is still in its prototype phase and has not yet been subjected to long-term performance evaluations or full-scale deployment. As such, further development, testing, and validation are necessary before it can be considered for official institutional adoption.

Overall, this study concludes that TUKIB presents a viable solution for addressing the inefficiencies of the existing system and lays the groundwork for a more responsive, accessible, and streamlined service platform for the RRC. While the system has not yet been officially implemented and deployed, its development demonstrates how user-centered design, combined with modern web technologies, can significantly enhance institutional service delivery when adopted. Furthermore, the insights gained from the development of TUKIB may serve as a foundation for similar workflow automation systems in other research centers or academic institutions.

\section{Recommendations}

TUKIB is still in its early stages of development, and the developers believe and acknowledge that it could further be improved and maximized to its full potential. Building upon the gathered data and feedback from the RRC staff and potential clients, the following recommendations are proposed for the continued development and deployment of TUKIB. 

This section is divided into two parts: "System Recommendations" and "Implementation and Deployment Recommendations". These recommendations were identified during the course of development and testing of TUKIB, and are crucial in improving TUKIB's performance, quality of service, and user experience.

\subsection{System Recommendations}

This subsection discusses the limitations observed in the current version of the TUKIB system. It also outlines the necessary features and modifications identified during development and testing that would help improve the system.

\begin{itemize}
	
	\item \textbf{Allow Users to Reset Password}
	
	To improve application availability and ensure that users are always able to log into their accounts, it is imperative to implement a "Reset Password" feature to all user types. The developers suggest adding this feature on the login page through a "Reset Password" or "Forget Password" button that would allow users to reset their password using their registered emails.
	
	\item \textbf{Chatbot Expansion}
	
	Currently, the chatbot's training data is small and limited to basic information regarding the services of RRC. The chatbot has difficulty particularly in recognizing names of people accurately. It also often struggles to correctly interpret user inputs that contain numbers—particularly dates, phone numbers, or reference codes. Hence, the developers recommend expanding the training data to enhance the accuracy of the chatbot's responses. Moreover, the developers also recommend implementing Retrieval-Augmented Generation (RAG) to improve the quality of chatbot responses. While the current system uses the Rasa framework, which serves its purpose, transitioning to or integrating RAG can offer more dynamic and context-aware interactions. 
	
	\item \textbf{Automated Email Notifications}
	
	Implement automated email alerts to promptly inform users and approvers of key status updates, including chargeslip approvals, service request progress, and reservation confirmations. These notifications should be sent to users' registered email addresses to enhance communication efficiency and streamline the overall process.
	
	\item \textbf{Built-in Direct Messaging}
	
	There is currently no feature that allows users to send messages to each other in real-time. Introducing a built-in direct messaging feature within the system would facilitate faster collaboration and issue resolution, especially between clients and staff.
	
	\item \textbf{Enhance Security Features}
	
	Strengthen the system’s security by incorporating two-factor authentication (2FA), enforcing strong password policies, and ensuring all sensitive data is encrypted both in transit and at rest. These measures will enhance user data protection and compliance with security best practices.
	
	\item \textbf{Full UI Responsiveness}
	
	The system’s user interface has been implemented with responsive design principles and mostly adapts well across various devices and screen sizes. However, there are still inconsistencies in layout and component behavior, particularly on tablets and smaller screens. Further refinement is needed to ensure consistent user experience and full optimization of all UI elements across desktops, tablets, and smartphones.

	\item \textbf{Audit Logging and Activity Tracking}
	
	Implement detailed audit logs to monitor system usage and track critical user activities. This ensures traceability, aids in debugging, and strengthens accountability and system security.
	
	\item \textbf{Downloadable Tabular Data}
	
	To support data analysis, reporting, and administrative efficiency, the developers recommend implementing a feature that allows staff to download system data in tabular format. The feature should be able to support automated conversion of data into various document formats including.csv (comma-separated file), .xls,	and/or .pdf. These downloadable data should include key datasets such as user information, service records, and request logs, enabling more effective decision-making and streamlined reporting processes.
	
\end{itemize}

\subsection{Implementation and Deployment Recommendations}

This subsection outlines recommended features and measures that could significantly enhance TUKIB or support its deployment, but whose implementation or initiation falls outside the current scope of the study. These recommendations are intended for future consideration to further improve system performance, user satisfaction, and long-term sustainability.

\begin{itemize}
	
	\item \textbf{Load and Performance Testing}
	
	Due to technical and resource constraints, the developers were unable to perform extensive load and performance testing. It is therefore recommended that future efforts include rigorous testing under multi-user scenarios to assess and optimize system scalability, responsiveness, and stability.
	
	
	\item \textbf{Pre-deployment usability testing}
	
	Prior to full deployment, it is recommended to conduct structured usability testing with a representative sample of end users, including both clients and RRC staff. This will help identify any usability concerns, improve system interaction flows, and ensure the system meets user expectations.
	
	\item \textbf{Comprehensive User Training}
	
	Develop and deliver formal training sessions and materials to ensure that users fully understand how to navigate and utilize the system effectively. This is especially important for first-time users and administrative personnel.
	
\end{itemize}