%   Filename    : chapter_5.tex 
\chapter{Conclusion and Recommendations}
This chapter presents the conclusion of the study and offers practical recommendations for future enhancement and implementation of TUKIB.

\section{Conclusion}

With rapid technological advancements, adaptability and flexibility- especially within businesses and institutions- is a must. Failure to keep up with the advancements can render an institution inefficient and potentially lead to its failure.

To address this problem, the researchers developed TUKIB, a system designed to automate and streamline service delivery at the Regional Research Center (RRC). The system was built with a strong focus on understanding and fulfilling user requirements. The developers also focused on creating an intuitive and user friendly interface to ensure that the service delivery of RRC would be efficient, not only for the staff but for the clients as well. 

The development of TUKIB employed HTML5, Bootstrap, and JavaScript for the frontend, while PostgreSQL and ReactJS powered the backend. The chatbot feature was implemented using Rasa. Additionally, all interfaces and system features were planned deliberately and were improved incrementally with testing and feedback from stakeholders. 

Key findings from the system analysis, design activities, and valuable input from users were drawn. First, transitioning to an automated, centralized platform like TUKIB is essential for modernizing service management at the RRC, significantly enhancing efficiency and accuracy compared to existing manual processes. Second, the active involvement of stakeholders throughout the design process was crucial in ensuring the system effectively addresses the specific needs and operational context of both RRC staff and clients. Third, the successful development and implementation of TUKIB provide a valuable model for other research centers and institutions aiming to automate and optimize their service operations.

Moreover, the consultation process highlighted the importance of maintaining open communication channels between developers and stakeholders. Regular feedback sessions is vital in adapting the system to evolving user needs.

\section{Recommendations}

TUKIB is still in its early stages of development, and the developers believe
that it could further be improved and maximized to its full potential. Building upon the gathered data and feedback from the RRC staff and potential clients, the following recommendations are proposed for the continued development and deployment of TUKIB. 

This section is divided into two parts: "System Recommendations" and "External dependencies recommendations". These recommendations were identified during the course of development and testing of TUKIB, and are crucial in improving TUKIB's performance, quality of service, and user experience.

\subsection{System Recommendations}

This subsection discusses the limitations observed in the current version of the TUKIB system. It also outlines the necessary features and modifications identified during development and testing that would help improve the system.

\begin{itemize}
	
	\item \textbf{Allow Users to Reset Password}
	
	To improve application availability and ensure that users are always able to log into their accounts, it is imperative to implement a "Reset Password" feature to all user types. The developers suggest adding this feature on the login page through a "Reset Password" or "Forget Password" button that would allow users to reset their password using their registered emails.
	
	\item \textbf{Chatbot Expansion}
	
	Currently, the chatbot's training data is small and limited to basic information regarding the services of RRC. The developers recommend expanding this training data to enhance the accuracy of chatbot responses. Moreover, the developers also recommend implementing Retrieval-Augmented Generation (RAG) to improve the quality of chatbot responses. While the current system uses the Rasa framework, which serves its purpose, transitioning to or integrating RAG can offer more dynamic and context-aware interactions.
	
	\item \textbf{Automated Email Notifications}
	
	Implement automated email alerts to promptly inform users and approvers of key status updates, including chargeslip approvals, service request progress, and reservation confirmations. These notifications should be sent to users' registered email addresses to enhance communication efficiency and streamline the overall process.
	
	\item \textbf{Built-in Direct Messaging}
	
	There is currently no feature that allows users to send messages to each other in real-time. Introducing a built-in direct messaging feature within the system would facilitate faster collaboration and issue resolution, especially between clients and staff.
	
	\item \textbf{Enhance Security Features}
	
	Strengthen the system’s security by incorporating two-factor authentication (2FA), enforcing strong password policies, and ensuring all sensitive data is encrypted both in transit and at rest. These measures will enhance user data protection and compliance with security best practices.
	
	\item \textbf{Full UI Responsiveness}
	
	The system’s user interface has been implemented with responsive design principles and mostly adapts well across various devices and screen sizes. However, there are still inconsistencies in layout and component behavior, particularly on tablets and smaller screens. Further refinement is needed to ensure consistent user experience and full optimization of all UI elements across desktops, tablets, and smartphones.
	
	\item \textbf{Downloadable Tabular Data}
	
	To support data analysis, reporting, and administrative efficiency, the developers recommend implementing a feature that allows staff to download system data in tabular format. The feature should be able to support automated conversion of data into various document formats including.csv (comma-separated file), .xls,	and/or .pdf.
	
	 This functionality should include key datasets such as user information, service records, and request logs, enabling more effective decision-making and streamlined reporting processes.
	
	\item \textbf{Plan for Future Scalability}
	
	Design the system with future scalability in mind, allowing for the seamless integration of additional services such as online payments or expanded facility rental options as the needs of the RRC evolve.
	
\end{itemize}

\subsection{Implementation and Deployment Recommendations}

This subsection outlines recommended features and measures that could significantly enhance TUKIB or support its deployment, but whose implementation falls outside the current scope of the study. These recommendations are intended for future consideration to further improve system performance, user satisfaction, and long-term sustainability.

\begin{itemize}
	
	\item \textbf{Pre-deployment usability testing}
	
	Conduct rigorous usability testing with representative clients and staff before the full deployment of TUKIB to identify and address any potential usability issues or areas for improvement.
	
	\item \textbf{Comprehensive User Training}
	
	Develop and deliver formal training sessions and materials to ensure that users fully understand how to navigate and utilize the system effectively. This is especially important for first-time users and administrative personnel.
	
\end{itemize}